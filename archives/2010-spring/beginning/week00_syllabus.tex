\documentclass[letterpaper]{article}

\usepackage[pdftex,letterpaper=true]{hyperref}
\usepackage{url}

\title{Beginning System Administration}
\author{Syllabus}
\date{Spring 2010}

\begin{document}

\maketitle

\section*{Course Information}

\begin{description}
  \item[CCN] CS98-15: 26414, CS198-15: 26624
  
  \item[Facilitators] Sanjay Krishnan (\texttt{sanjayk+decal@ocf.berkeley.edu}) and Alan Wong (\texttt{alanw+decal@ocf.berkeley.edu})
  
  \item[Lab Assistant] Jordan Salter (\texttt{jordan+decal@ocf.berkeley.edu})
  
  \item[Office Hours] \url{http://www.ocf.berkeley.edu/staff_hours} and by appointment.
  
 % \item[Location and Time] 310 Soda, Monday 6-7pm; 271 Soda Monday 7-8pm
 
 \item[Time] Monday 6-8pm
 
 \item[Location] 310 Soda 6-7pm (Lecture), 271 Soda 7-8pm (Lab)
  
  \item[Website] \url{http://www.ocf.berkeley.edu/sysadmin-decal/beginning}
  
  \item[Course Description] 
  The course will cover the setup and administration of a
	production-quality web server. Topics include the Internet
	infrastructure, use of a Unix environment, database use and
	administration, security in a multi-user environment, and emerging
	web technologies.
	
  \item[Course Goals]
		After completing this course, students will be able to setup and
		secure their own web server from a minimal installation of a
		GNU/Linux installation. Students will also be able to coordinate and
		administer a multi-user web application environment.

  \item[Grading]
		This course will be graded on a Passed/Not Passed basis. For the
		purposes of this course, a Pass will be considered at least 60\%
		correct. The final grade will be determined as follows: 50\%
		project, 30\% laboratories, and 20\% homework. 

		To obtain a `Pass' grade in this course, students \textbf{must
		complete} a final project, turn in \emph{all} assigned homework,
		and submit laboratory reports for \emph{all} assigned laboratories.
		Up to \textbf{3} laboratories and/or homework assignments may be dropped; the
		final project may not be dropped.  \textbf{Failure to complete the
		final project will result in an automatic NP!}
	
		Laboratories and homework will be assigned during each course
		meeting and must be completed prior to the next course meeting.

  \item[Laboratory] 
    Laboratories are in-class exercises done in a computer lab that are
		designed to help students familiarize themselves with all the
		material covered up to that week's lecture. This hands-on approach
		is one of the most useful aspects of this class as it allows
		students to apply concepts and ideas learned from lectures and books
		as opposed to merely `book-learning.' A laboratory write-up (answers
		to specific questions) is to be turned in for each corresponding
		laboratory.
		
  \item[Homework]
		Homework are assignments that allow students to apply
		problem-solving skills and go more in-depth into the material being
		covered by giving specific scenarios and problems that may arise
		during their setup and asking students how they would respond or
		what kind of actions they would take in response. Assigned on a
		weekly basis, the format will generally be multiple-choice, fill in
		the blank, or require paragraph-length responses.

  \item[Final Project]
		The specifications for the final project is TBD, but
		previous projects done in prior semesters include: setting up and
		securing a multi-user web environment and looking for
		vulnerabilities on other students' project while protecting your own
		server, building and maintaining an Internet server, and designing
		and implementing different features and utilities not covered in class that benefit users, system administrators, or everyone alike.

\end{description}

\section*{Tentative Schedule}

\begin{tabular*}{\textwidth}[l]{r|l}
Week & Topic \\
\hline\hline
1 & Introduction to System Administration \\[0.5em]
2 & Introduction to UNIX \\[0.5em]
3 & UNIX Commands \\[0.5em]
4 & The Internet \\[0.5em]
5 & Server Daemons \\[0.5em]
6 & The Simplification of Week 5 \\[0.5em]
7 & LAMP \\[0.5em]
8 & Multi-User Environments \\[0.5em]
9 & Security \\[0.5em]
10 & Special Topics: TBD \\[0.5em]
11 & Special Topics: TBD \\[0.5em]
%12 & Final Project\\[0.5em]
12 & Final Project Presentations\\[0.5em]
\end{tabular*}

\end{document}


%\subsection{Beginning}
%
%\begin{tabular}[l]{l l l}
%Week & Topic (Beginning) & Topic (Intermediate) \\
%Week 1 & Introduction to System Administration \\
%Week 2 & Introduction to UNIX\\
%Week 3 & UNIX Commands\\
%Week 4 & The Internet\\
%Week 5 & Server Daemons\\
%Week 6 & The Simplification of Week 5\\
%Week 7 & LAMP\\
%Week 8 & Multi-User Environments\\
%Week 9 & Security\\
%Week 10 & Student Project / Special Topics\\
%Week 11 & Student Project / Special Topics\\
%\end{tabular}
%
%\subsection{Intermediate}
%
%\begin{tabular}[l]{l l}
%Week 1 & Introduction to System Administration \\
%Week 2 & Introduction to UNIX\\
%Week 3 & UNIX Commands\\
%Week 4 & The Internet\\
%Week 5 & Server Daemons\\
%Week 6 & The Simplification of Week 5\\
%Week 7 & LAMP\\
%Week 8 & Multi-User Environments\\
%Week 9 & Security\\
%Week 10 & Student Project / Special Topics\\
%Week 11 & Student Project / Special Topics\\
%\end{tabular}
