%% LyX 1.3 created this file.  For more info, see http://www.lyx.org/.
%% Do not edit unless you really know what you are doing.
\documentclass[english]{article}
\usepackage[T1]{fontenc}
\usepackage[latin1]{inputenc}

\makeatletter

%%%%%%%%%%%%%%%%%%%%%%%%%%%%%% LyX specific LaTeX commands.
%% Because html converters don't know tabularnewline
\providecommand{\tabularnewline}{\\}

%%%%%%%%%%%%%%%%%%%%%%%%%%%%%% Textclass specific LaTeX commands.
 \newenvironment{lyxlist}[1]
   {\begin{list}{}
     {\settowidth{\labelwidth}{#1}
      \setlength{\leftmargin}{\labelwidth}
      \addtolength{\leftmargin}{\labelsep}
      \renewcommand{\makelabel}[1]{##1\hfil}}}
   {\end{list}}
 \newenvironment{lyxcode}
   {\begin{list}{}{
     \setlength{\rightmargin}{\leftmargin}
     \setlength{\listparindent}{0pt}% needed for AMS classes
     \raggedright
     \setlength{\itemsep}{0pt}
     \setlength{\parsep}{0pt}
     \normalfont\ttfamily}%
    \item[]}
   {\end{list}}

\usepackage{babel}
\makeatother
\begin{document}

\title{System Administration for the Web:\\
Week 7 Lecture Notes\\
Basic Programming and Perl}


\date{October 24, 2005}

\maketitle

\section{What is Programming?}

Most computer scientists would define programming as implementing
an abstract algorithm using a programming language to produce a set
of instructions that a computer can understand.

I think that's a dull and very difficult to understand definition.
Programs are like recipes for making food, and, using that analogy,
programming is the process of organizing the production of a certain
type of food into a set of consistent and reproducible steps.

Some recipes are quite simple and linear, like those for mixing drinks.
You take a certain amount of various ingredients and combine them
in a certain order. Other recipes are more complicated and usually
require making decisions: to flip or not to flip a steak on a grill
or whether to take cookies out of an oven.

Regardless of the complexity of a certain computer program, though,
programming is almost always accomplished in 4 steps:

\begin{enumerate}
\item Planning
\item Prototyping and Coding
\item Testing and Debugging
\item Release
\end{enumerate}
Step 2 is often repeated after Step 3 upon the discovery of programming
errors.


\section{Why Program?}

Simple: \emph{fame}, \emph{fortune}, or \emph{laziness}.\\
In the context of system administration, programming usually falls
under laziness. As you've seen with shell scripts, programs can automate
repetitive tasks and save system administrators a ton of work.


\section{Perl}

The programming language you will be learning is Perl, the Practical
Extraction and Reporting Language. Perl was originally created by
Larry Wall, a one-time graduate student at UC Berkeley, and is currently
maintained by programmers all over the world. Perl was originally
designed to manipulate text files such as log files -- hence \emph{extraction}
and \emph{reporting} -- but has evolved into an extremely-versatile
language for everything from shell scripts to corporate websites.


\subsection{Perl's Strengths}

\begin{itemize}
\item Excellent text manipulation
\item Fast prototyping
\item Flexible syntax
\item No artificial limits
\end{itemize}

\subsection{Perl's Weaknesses}

\begin{itemize}
\item Interpreted language
\item Flexible syntax
\end{itemize}

\subsection{How is Perl Used?}

\begin{itemize}
\item Amazon.com
\item MovableType
\item Slashdot
\item Sweden's pension system
\item A server near you
\end{itemize}

\section{Data and Data Structures}

Data is anything your program must track, be it user-input, dates,
values, quantities, colors, etc. \emph{Scalar data} is the basic unit
of data in Perl. 

Scalar data can be anything from letters, words, numbers, sentences,
paragraphs, punctuation marks, or any combination of these categories.
To Perl, scalar data is a merely a stream of characters, such as letters,
numbers, spaces, and symbols. Additionally, there is a category of
characters that require special representation in Perl through the
use of \emph{escape-sequences}; common examples of such characters
are found in Table \ref{tab:Common-Perl-Escape-Sequences}.

Scalar data in Perl is held in three types of basic containers or
data structures: \emph{variables}, \emph{arrays}, and \emph{hashes}.

%
\begin{table}
\begin{center}\begin{tabular}{|c|c|}
\hline 
Escape-Sequence&
Meaning\tabularnewline
\hline
\hline 
\texttt{\textbackslash{}n}&
Newline\tabularnewline
\hline 
\texttt{\textbackslash{}t}&
Tab\tabularnewline
\hline 
\texttt{\textbackslash{}\textbackslash{}}&
Backslash (\texttt{\textbackslash{}})\tabularnewline
\hline 
\texttt{\textbackslash{}''}&
Double quote (\texttt{{}``})\tabularnewline
\hline
\end{tabular}\end{center}


\caption{\label{tab:Common-Perl-Escape-Sequences}Common Perl Escape-Sequences}
\end{table}
 

\begin{description}
\item [NOTE:]Perl is a very context-sensitive language. This means that
it will try to automatically understand how you wish to use a piece
of scalar data. For example, the scalar data {}``\texttt{45}'' could
be interpreted as either the number 45 or a string composed of the
characters 4 and 5. Unlike other programming languages, where you
much always specify the context under which you are using a piece
of scalar data, Perl will automatically decide for you depending on
what you do to the data.
\end{description}

\subsection{Variables}

Variables are the simplest type of data structure in Perl; they hold
a single piece of scalar data. Variable names begin with a dollar
sign (\texttt{\$}) followed by characters that uniquely identify each
variable -- \texttt{\$sample\_variable}.


\subsubsection*{Assignment}

Assigns a piece of scalar data to a variable

\begin{lyxlist}{00.00.0000}
\item [Operator:]\texttt{=}
\item [Example:]\texttt{\$sample\_variable = {}``sdf343fd'';}
\item [Result:]\texttt{\$sample\_variable = {}``sdf343fd''}
\end{lyxlist}

\subsubsection*{Concatenation}

Joins pieces of scalar data together

\begin{lyxlist}{00.00.0000}
\item [Operator:]\texttt{.}
\item [Example:]\texttt{\$sample\_variable = {}``year'' . {}``2005'';}
\item [Result:]\texttt{\$sample\_variable = {}``year2005''}
\end{lyxlist}

\subsubsection*{Chomp\label{sub:Chomp}}

Removes a newline from the end of a piece of scalar data

\begin{lyxlist}{00.00.0000}
\item [Operator:]\texttt{chomp}
\item [Example:]\texttt{\$sample\_variable = {}``This a sentence.\textbackslash{}n'';}~\\
\texttt{chomp \$sample\_variable;}
\item [Result:]\texttt{\$sample\_variable = {}``This is a sentence.''}
\end{lyxlist}

\subsubsection*{Mathematical Functions}

Performs mathematical operations on numeric scalar data\\
Note: Addition is the example presented, but other mathematical operators
(\texttt{{*}}, \texttt{/}, \texttt{-}) work too.

\begin{lyxlist}{00.00.0000}
\item [Operator:]\texttt{+}
\item [Example:]\texttt{\$sample\_variable = {}``10'';}~\\
\texttt{\$another\_variable = \$sample\_variable + 1;}
\item [Result:]\texttt{\$sample\_variable = {}``10''}~\\
\texttt{\$another\_variable = {}``11''}
\end{lyxlist}

\subsection{Arrays}

Arrays are data structures that hold multiple pieces of scalar data,
each \emph{indexed} by a number. Arrays are best described as a numbered
list of scalar data pieces. Array names begin with an at-sign (\texttt{@})
followed by characters that uniquely identify each array -- \texttt{@sample\_array}.
Elements of an array are referenced by a dollar sign followed by the
name of the array and the index of the element in square brackets
-- \texttt{\$sample\_array{[}0{]}}. Indices numbering begins at zero.

\begin{description}
\item [NOTE:]Most variable operations can be performed on individual array
elements.
\end{description}

\subsubsection*{Assignment Method \#1}

Assigns scalar data to an array

\begin{lyxlist}{00.00.0000}
\item [Operator:]\texttt{=}
\item [Example:]\texttt{\$sample\_array{[}0{]} = {}``data\_piece\_0'';}~\\
\texttt{\$sample\_array{[}1{]} = {}``data\_piece\_1'';}
\item [Result:]\texttt{@sample\_array =}~\\
\texttt{0 -> {}``data\_piece\_0''}~\\
\texttt{1 -> {}``data\_piece\_1''}
\end{lyxlist}

\subsubsection*{Assignment Method \#2}

Assigns scalar data to an array

\begin{lyxlist}{00.00.0000}
\item [Operator:]\texttt{()}
\item [Example:]\texttt{@sample\_array = ({}``data\_piece\_0'', {}``data\_piece\_1'');}
\item [Result:]\texttt{@sample\_array =}~\\
\texttt{0 -> {}``data\_piece\_0''}~\\
\texttt{1 -> {}``data\_piece\_1''}
\end{lyxlist}

\subsubsection*{Pop}

Removes the last element of an array and returns it

\begin{lyxlist}{00.00.0000}
\item [Operator:]\texttt{pop}
\item [Example:]\texttt{@sample\_array = ({}``data\_piece\_0'', {}``data\_piece\_1'');}~\\
\texttt{\$temp = pop @sample\_array;}
\item [Result:]\texttt{@sample\_array =}~\\
\texttt{0 -> {}``data\_piece\_0''}~\\
\texttt{\$temp = {}``data\_piece\_1''}
\end{lyxlist}

\subsubsection*{Push}

Adds an element to the end of an array

\begin{lyxlist}{00.00.0000}
\item [Operator:]\texttt{push}
\item [Example:]\texttt{@sample\_array = ({}``data\_piece\_0'', {}``data\_piece\_1'');}~\\
\texttt{push @sample\_array, {}``data\_piece\_2'';}
\item [Result:]\texttt{@sample\_array =}~\\
\texttt{0 -> {}``data\_piece\_0''}~\\
\texttt{1 -> {}``data\_piece\_1''}~\\
\texttt{2 -> {}``data\_piece\_2''}
\end{lyxlist}

\subsubsection*{Shift and Unshift}

Removes and adds an element to the beginning of an array, respectively\\
Similar syntax as \texttt{pop} and \texttt{push} 


\subsection{Hashes}

Hashes are the most flexible basic data structure available in Perl.
They are similar to arrays in that they hold multiple pieces of scalar
data, but rather than having elements indexed by a number, they are
indexed by words or, more appropriately, \emph{keys}. Hash names begin
with a percentage sign (\texttt{\%}) followed by characters that uniquely
identify each hash -- \texttt{\%sample\_hash}. Elements of an array
are referenced by a dollar sign followed by the name of the array
and the key of the element in curly brackets and quotation marks --
\texttt{\$sample\_hash\{''key\_name''\}}.

\begin{description}
\item [NOTE:]Most variable operations can be performed on individual hash
elements.
\end{description}

\subsubsection*{Assignment Method \#1}

Assigns scalar data to a hash

\begin{lyxlist}{00.00.0000}
\item [Operator:]\texttt{=}
\item [Example:]\texttt{\$sample\_hash\{''name''\} = \char`\"{}stephen\char`\"{};}~\\
\texttt{\$sample\_hash\{''location''\} = \char`\"{}berkeley\char`\"{};}
\item [Result:]\texttt{\%sample\_hash =}~\\
\texttt{{}``name'' -> \char`\"{}stephen\char`\"{}}~\\
\texttt{{}``location'' -> \char`\"{}berkeley\char`\"{}}
\end{lyxlist}

\subsubsection*{Assignment Method \#2}

Assigns scalar data to a hash

\begin{lyxlist}{00.00.0000}
\item [Operator:]\texttt{=, =>}
\item [Example:]\texttt{\%sample\_hash = (}~\\
\texttt{{}``name'' => \char`\"{}stephen\char`\"{},}~\\
\texttt{{}``location'' => {}``berkeley'',}~\\
\texttt{);}
\item [Result:]\texttt{\%sample\_hash =}~\\
\texttt{{}``name'' -> \char`\"{}stephen\char`\"{}}~\\
\texttt{{}``location'' -> \char`\"{}berkeley\char`\"{}}
\end{lyxlist}

\subsubsection*{Keys}

Returns an array of the keys of a hash (in no particular order)

\begin{lyxlist}{00.00.0000}
\item [Operator:]\texttt{keys}
\item [Example:]\texttt{\%sample\_hash = (}~\\
\texttt{{}``name'' => \char`\"{}stephen\char`\"{},}~\\
\texttt{{}``location'' => {}``berkeley'',}~\\
\texttt{);}~\\
\texttt{@my\_array = keys \%sample\_hash;}
\item [Result:]\texttt{@my\_array =}~\\
\texttt{0 -> {}``location''}~\\
\texttt{1 -> {}``name''}
\end{lyxlist}

\subsection{Advanced Data Structures}

It is possible to \emph{nest} one type of data structure inside of
another data structure or even inside the same type of data structure;
that is, it is possible to create arrays of hashes or hashes of hashes.
However, the syntax and theory for doing so is beyond the scope of
this Perl introduction. Please refer to the command \texttt{perldoc
perlreftut} or any other documentation on Perl \emph{references}.


\section{Control Structures and Comparison Operators}

Sometimes it is necessary to make a decision in a program or to perform
a set of actions multiple times. \emph{Control structures} provide
the facility to do both of these things.

Control structures are blocks of Perl code enclosed in curly brackets
whose execution is dependent on some kind of \emph{test(s)}. These
tests are usually performed using \emph{comparison operators} to make
some sort of comparison, but they can also be the value of some variable.
The most commonly used control structures in Perl are \texttt{if},
\texttt{while}, and \texttt{foreach}.


\subsection{The if Control Structure}

The basic structure of an \texttt{if} control structure is:

\begin{lyxcode}
if~(TEST)~\{

~~~~~CODE\_IF\_TRUE

\}
\end{lyxcode}
where \texttt{TEST} is some sort of comparison or value, that, if
true, results in the execution of \texttt{CODE\_IF\_TRUE}, which can
be multiple Perl operations.


\subsection{Comparison Operators}

Perl provides a group of comparison operators that can be used in
test conditions for control structures. Comparison operators are divided
into two categories, numeric and string operators (Table \ref{tab:Numeric-and-String}).
It is important that you use the correct type of comparison operator;
using the improper type of comparison operator will result in unexpected
program behavior. 

%
\begin{table}
\begin{center}\begin{tabular}{|c|c|c|}
\hline 
Comparison&
Numeric&
String\tabularnewline
\hline
\hline 
Equal&
\texttt{==}&
\texttt{eq}\tabularnewline
\hline 
Not equal&
\texttt{!=}&
\texttt{ne}\tabularnewline
\hline 
Less than&
\texttt{<}&
\texttt{lt}\tabularnewline
\hline 
Greater than&
\texttt{>}&
\texttt{gt}\tabularnewline
\hline 
Less than or equal to&
\texttt{<=}&
\texttt{le}\tabularnewline
\hline 
Greater than or equal to&
\texttt{>=}&
\texttt{ge}\tabularnewline
\hline
\end{tabular}\end{center}


\caption{\label{tab:Numeric-and-String}Numeric and String Comparison Operators}
\end{table}


I mentioned earlier that test conditions could also be values. In
Perl, any value other than \texttt{0} or \emph{null} (an undefined
variable) is true. Consequently, you can provide a variable name as
a test condition, and, depending upon its value or whether it's defined,
the test will return true or false.


\subsection{The while Control Structure}

The basic structure of a \texttt{while} control structure is:

\begin{lyxcode}
while~(TEST)~\{

~~~~~CODE\_IF\_TRUE

\}
\end{lyxcode}
where \texttt{TEST} is some sort of comparison or value, that, if
true, results in the execution of \texttt{CODE\_IF\_TRUE}, which can
be multiple Perl operations. \texttt{CODE\_IF\_TRUE} will be repeated
until \texttt{TEST} is false, so it is important that \texttt{CODE\_IF\_TRUE}
does something to modify the test condition so that the program does
not execute \texttt{CODE\_IF\_TRUE} forever -- creating an infinite
loop.


\subsection{The foreach Control Structure}

The basic structure of a \texttt{foreach} control structure is:

\begin{lyxcode}
foreach~\$TEMP\_VAR~(@ARRAY)~\{

~~~~~CODE\_PER\_ARRAY\_ELEMENT

\}
\end{lyxcode}
where \texttt{\$TEMP\_VAR} is the name of a temporary variable that
will hold an element of \texttt{@ARRAY}. foreach will copy an element
of \texttt{@ARRAY} into \texttt{\$TEMP\_VAR}, and execute \texttt{CODE\_PER\_ARRAY\_ELEMENT}
for each -- hence the name -- element of array. \texttt{foreach} control
structures are the best way to perform some action on each element
of an array. 


\section{Basic Input and Output}


\subsection{Input}

You can get input from a user by using the assigning the value \texttt{<STDIN>}
to a variable. For example,

\begin{lyxcode}
\$the\_user\_input~=~<STDIN>;
\end{lyxcode}
which will read the input from a user until the Return key is pressed.

\begin{description}
\item [NOTE:]The value assigned to such a variable will also contain the
newline character, since the input is terminated by a Return. Consequently,
it is advisable to \texttt{chomp} the variable (Section \ref{sub:Chomp}).
\end{description}

\subsection{Output}

You can present output to a user by using the \texttt{print} command.
For example,

\begin{lyxcode}
print~{}``Hello~World!\textbackslash{}n'';
\end{lyxcode}
which will display the string {}``Hello World!'' followed by a newline.


\section{A Simple Perl Program}

\begin{lyxcode}
\#!/usr/bin/perl

print~\char`\"{}I'm~thinking~of~a~number~between~1~and~100.\textbackslash{}n\char`\"{};

\$guess~=~-1;

\$number~=~int~rand(100)~+~1;

while~(\$guess~!=~\$number)~\{

~~~~~print~\char`\"{}What~is~your~guess?~\char`\"{};

~~~~~\$guess~=~<STDIN>;

~~~~~chomp~\$guess;

~~~~~if~(\$guess~<~\$number)~\{~print~\char`\"{}Too~low!\textbackslash{}n\char`\"{};~\}

~~~~~if~(\$guess~>~\$number)~\{~print~\char`\"{}Too~high!\textbackslash{}n\char`\"{};~\}

\}

print~\char`\"{}You~got~it~right!\textbackslash{}n\char`\"{};
\end{lyxcode}

\end{document}
